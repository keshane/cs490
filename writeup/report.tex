%        File: report.tex
%     Created: Thu May 05 04:00 AM 2016 E
% Last Change: Thu May 05 04:00 AM 2016 E
%
\documentclass[letterpaper]{article}
\author{Keshane Gan\\
        Advisor: James Aspnes\\
        CPSC 490\\
        Department of Computer Science\\
        Yale University}
\date{May 5, 2016}
\title{Assigning and Scheduling Teachers and Students}
\begin{document}
\maketitle
\section{Introduction and Motivation}
Every school year, the Yale University Guild of Carillonneurs recruit new members to join their group to play a musical
instrument called the carillon. During this process, each member of the Guild teaches several students individually how to play the
carillon (through a process they call Heel) before those students audition for membership. These lessons are scheduled
once a week over nine weeks. Since the founding of the Guild, the responsibility of scheduling these lessons lay in the
hands of one Guild member. 

The popularity of the Guild has made this responsibility more time consuming and stressful. Several hours are usually
spent with one or two additional helpers trying to schedule teachers and students. Several hard [H] and soft [S] constraints
must be satisfied:
\begin{itemize}
    \item {[H]} Both the teacher and the student have to be available for the lesson time.
    \item {[H]} Because there is only one practice instrument available, only one lesson may be taking place at a time.
    \item {[S]} Each lesson is 30 minutes long and are assigned to half-hour slots between 8:00 AM and 12:00 AM. For various
        reasons, some of these slots may have to be made unavailable for lessons.
    \item {[H]} Students must be taught by teachers at least one college class year higher. For example, freshman may be
        taught by a sophomore, junior, or senior, but a sophomore can only be taught by a junior or senior. For the
        cases when a graduate or professional school student is taking lessons, a Guild member also in the graduate or
        professional schools should teach the student. The Guild may not have such a member, so a college senior may
        also be the teacher.
    \item {[H]} Students must not be taught by a Guild member whom he or she personally knows.
    \item {[S]} Teachers should be given students with a variety of musical experiences.
    \item {[S]} All teachers should have roughly the same number of students.
\end{itemize}

A couple of other important considerations is that the number of members in the Guild has historically been between 15
and 20 and the number of students in Heel has been between 50 and 100. Automating the scheduling process would allow the
responsible Guild members to focus their energy and efforts on other important aspects of coordinating Heel. I aim to
provide them that automation. Note: for brevity and clarity, members of the Guild who act as the teachers will be referred to
as \textit{Guildies} and the students who are taking lessons to learn carillon will be called \textit{Heelers}.

\section{Background}

\end{document}


